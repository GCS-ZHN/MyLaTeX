\section{数学公式}
本部分许多内容需要amsmath宏包,请用\textbackslash usepackage\{amsmath\}导入。
\subsection{行内公式}
众所周知,markdown中的行内公式是使用一对\$来界定,而这个习俗正是来自于latex原生语法规则。
例如勾股定理公式为$a^2+b^2=c^2$。其公式语法由众多特定标记形成。

\subsection{行间公式}
在markdown中行间公式由一对\$\$来界定,而latex中直接使用equation环境来界定。
例如公式\ref{weijifen}是牛顿-莱布尼兹公式:
\begin{equation}
    \int_a^bf(x)dx=F(b)-F(a) \label{weijifen}
\end{equation}

此外可以使用\textbackslash [和\textbackslash ]包裹进行无编号的行间公式编写,例如下面公式是
余弦定理的表述,此时\textbackslash label命令不显示标签:
\[c^2=a^2+b^2-2ab\cos (C) \]

\subsection{多行公式}
\subsubsection{长公式换行}
有的时候一个公式很长,需要进行换行,可以使用multline环境,使用\textbackslash\textbackslash进行换
行。公式编号居末尾,例如:

\begin{multline}
a + b + c + d + e +f + g + h +i \\
= j + k + l + +m +n \\
= o +p + q + r + s \\
= t + u + v + x + z 
\end{multline}

\subsubsection{多行多公式}
例如需要多行多个公式,等号对齐,此时可以使用align和gather等环境,align可以将\&两侧的部分对齐,例如:
\begin{align}
    a & = b + c \\
    & = d +e
\end{align}

而gather则不会对齐
\begin{gather}
    a = b + c \\
    d = e + f + g \\
    h + i = j + k \notag \\
    l + m = n
\end{gather}